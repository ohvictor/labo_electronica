\chapter{Medicion de distorción armónica}
\section{Medición}
Utilizando el analizador de espectros, se midió la distorsión armónica del generador de funciones Agilent(modelo)
con una señal senoidal de $0.7\si{\mega\hertz}$ y $250\si{\milli\volt} pp$.

Para calcular la distorsión armónica total (THD) medida con el analizador, se utilizaron las ecuaciones \ref{eq:THD}
y \ref{eq:P_k}.
\begin{equation}
    THD=\frac{\sum_{j=1}^{n} P_j}{\sum_{i=0}^{n} P_i}
    \label{eq:THD}
\end{equation}

\begin{equation}
    P_k[\si{\milli\watt}]= 1\si{\milli\watt} * 10^{P_k[dBm]/10}
    \label{eq:P_k}
\end{equation}

Entonces,

\begin{equation*}
    P_0 = 123 mW;P_1 = 123 mW;P_2 = 123 mW
\end{equation*}

\begin{equation}
    \Rightarrow THD = Ans
\end{equation}

Con las mediciones y cácluclos anteriores y utilizando otros generadores de funciones, se obtuvo la siguiente tabla:

\begin{table}[h]
    \begin{center}
        \begin{tabular}{|c|c|c|c|c|c|c|c|}
            \hline
            Modelo & $P_0 (mW)$ & $P_1 (mW)$ & $P_2(mW)$ & $P_3(mW)$ & $P_4(mW)$ & $THD$ & $THD_{Fab}$ \\
            \hline
            Agilent  & & & & & & & \\
            Picotest & & & & & & & \\
            Instek   & & & & & & & \\
            \hline
        \end{tabular}
    \end{center}
\end{table}

\section{Comparación con la hoja de datos}

\section{Conclusiones}