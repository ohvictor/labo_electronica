\pgfplotstableset{
    alias/nom/.initial = Nominal,
    alias/obs/.initial = Observaciones,
}
\begin{table}[h]
    \begin{center}
        \pgfplotstabletypeset[
            %General%
            col sep = comma,
            string type,
            columns = {nom,L1,Q1,P1,L10,Q10,P10,L100,Q100,P100,obs},
            every head row/.style={
                before row = {
                    \toprule
                    Valor & \multicolumn{3}{c|}{$f = 1\si{k\hertz}$} & \multicolumn{3}{c|}{$f = 10\si{k\hertz}$} & \multicolumn{3}{c|}{$f = 100\si{k\hertz}$} &\\
                },
                after row = {\midrule}
            },
            column type/.add = {|}{},
            every last column/.style={
                column type/.add={}{|},
            },
            every last row/.style={after row=\bottomrule},
            %Especifico cada columna%
            columns/nom/.style={
                column name = Nominal,
            },
            columns/obs/.style={
                column name = Observaciones,
            },
            columns/L1/.style={
                column name = $L$,
            },
            columns/Q1/.style={
                column name = $Q$,
            },
            columns/P1/.style={
                column name = $P$,
            },
            columns/L10/.style={
                column name = $L$,
            },
            columns/Q10/.style={
                column name = $Q$,
            },
            columns/P10/.style={
                column name = $P$,
            },
            columns/L100/.style={
                column name = $L$,
            },
            columns/Q100/.style={
                column name = $Q$,
            },
            columns/P100/.style={
                column name = $P$,
            },
            ]{tabla3d.csv}
        \caption{Tabla Rellenable Ejercicio 3}
        \label{tab:test1}
    \end{center}
\end{table}