%Preambulo

\documentclass{article}
    \usepackage{pgfplots}
    \usepackage{pgfplotstable}
    \pgfplotsset{compat=1.16}
    \usepackage{booktabs}
    \usepackage{siunitx}
    
        \pgfplotstableset{
            alias/fgen/.initial = Frecuencia Generador,
            alias/var1/.initial = Componente Variable 1,
            alias/var2/.initial = Componente Variable 2,
            alias/fcal/.initial = Frecuencia Calculada,
        }
    
    %Documento (sacando el environment document)
    
        \begin{document}
            Placeholder text.
    
            \begin{table}[h]
                \begin{center}
                    \pgfplotstabletypeset[
                        col sep = comma,
                        columns={fgen, var1, var2, fcal, Error},
                        column type/.add = {|}{},
                        columns/fgen/.style={
                            column name = $f_{GEN}$,
                            sci, sci zerofill, precision = 3                        
                        },
                        columns/var1/.style={
                            column name = $Z_1$,
                        },
                        columns/var2/.style={
                            column name = $Z_2$,
                        },
                        columns/fcal/.style={
                            column name = $f_{CAL}$,
                            sci, sci zerofill, precision = 3
                        },
                        every last column/.style={
                            column type/.add = {}{|},
                        },
                        every head row/.style={
                            before row = {\toprule},
                            after row = {
                                $[\si{\hertz}]$ & & & $[\si{\hertz}]$ & $[\%]$\\
                                \midrule}
                        },
                        every last row/.style={after row=\bottomrule},
                        ]{tabla2b.csv}
                    \caption{Tabla Rellenable Ejercicio 2}
                    \label{tab:test1}
                \end{center}
            \end{table}
    
            Exercise 2 Table has been completed for automation in Table \ref{tab:test1}.
            
    
        \end{document}