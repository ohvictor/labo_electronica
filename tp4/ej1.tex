\chapter{Métodos de medición de puentes}

La principal dificultad en la elección del método de medición de puentes 
está en que hay que medir tensiones diferenciales de muy baja magnitud. 
Estas tensiones idealmente deben ser cero (lo cual en la practica no es 
realizable), haciendo que el ruido que puedan tener los instrumentos sea 
un problema importante. 

\section{Osciloscopio}

Un osciloscopio trae problemas a la hora de medir, debido a que tiene un 
piso de ruido, lo que dificulta las mediciones en tensiones bajas, que son 
las que hay que medir en un puente. Otro problema es que para poder medir 
esto, que es una tensión diferencial, no se puede utilizar una sola punta, 
debido a que esta tiene que estar vinculada a la tierra del circuito. Esto 
obliga a conectar dos puntas y luego hacer la resta de las señales.

La ventaja que tiene este instrumento, es que se puede medir la tensión 
diferencial en funcion del tiempo, lo que permite apreciar su componente 
en magnitud y fase.

\section{Multímetro de precisión}

El multímetro de precisión facilita la medición debido a que se puede 
hacer una medición de la tensión diferencial de forma directa, disminuyendo 
el error de comparación del osciloscopio. La desventaja es que no se puede medir 
la tensión diferencial en función del tiempo.

\section{Amplificador de Instrumentación}

Un amplificador de instrumentación posee un elevado coeficiente de rechazo 
de modo común que minimiza mucho el ruido, lo cual, conectándolo a un 
osciloscopio, hace que se elimine el piso de ruido de este. Esto hace las 
mediciones más acertadas y más confiables. Mientras más alto sea el CMRR 
(cociente entre ganancia en modo diferencial y modo común) mejor resultado 
se va a obtener a la salida.