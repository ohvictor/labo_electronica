\documentclass{article}
\usepackage{pgfplotstable}
\usepackage{siunitx}
\usepackage{booktabs}

    \sisetup{
        round-mode      = places,
        round-precision = 2,
    }

    \begin{document}
        Placeholder text.

        \begin{table}[h]
            % \pgfplotstabletypeset[
            %     col sep = comma,
            %     display columns/0/.style = {
            %         column name = $f_{GEN}$,
            %         column type ={S},string type
            %     }
            %     display columns/1/.style = {
            %         column name = $Z_1$,
            %         column type ={S},string type
            %     }
            %     display columns/2/.style = {
            %         column name = $Z_2$,
            %         column type ={S},string type
            %     }
            %     display columns/3/.style = {
            %         column name = $f_{CAL}$,
            %         column type ={S},string type
            %     }
            %     display columns/4/.style = {
            %         column name = $Error$,
            %         column type ={S},string type
            %     }
            %     every head row/.style={
            %         before row ={\toprule},
            %         after row ={\midrule},
            %     }
            %     every last row/.style={
            %         after row=\bottomrule,
            %     },
            % ]{tabla2b.csv}
            \pgfplotstabletypeset[col sep = comma]{tabla2b.csv}
            \caption{Autogenerated table from .csv file}
            \label{table1}
        \end{table}

    \end{document}