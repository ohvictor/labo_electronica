\chapter{Medición de Inductancias}
    \section{Introducción}
    \label{sec:ej3Intro}
    \begin{figure}[h]
        \begin{center}
            \begin{circuitikz}[scale = 0.75, transform shape]
    \draw
    (0,0)
    to [sV, l=$V_g$] (0,8)
    to (4,8)
    to [R, l=$R_1$] (4,4)
    to [C, l=$C_3$] (4,2)
    to [R, l=$R_3$] (4,0) to (0,0)
    (4,8) to (8,8)
    to [L, l=$L_x$] (8,4)
    to [R, l=$R_4$] (8,0) to (4,0)
    (4,4) to [voltmeter, l=$V_d$] (8,4)
    (8,7) to (7,7) to [R, l=$R_x$] (7,5) to (8,5)
    ;
\end{circuitikz}
\caption{Puente de Hay}
\label{fig:Hay}
        \end{center}
        
    \end{figure}
    En principio, dado que la bobina entregada tiene un factor de calidad
    $Q_N\approx53$ a una frecuencia $f=10\si{\kilo\hertz}$, se decidió utilizar un
    puente de Hay, como visto en la Figura \ref{fig:Hay}, en lugar de un puente de
    Maxwell, el cual funciona apropiadamente en un rango de $Q\in(1,10)$.
    
    \section{Especificaciones}
    \label{sec:ej3Specs}
    Se desea que el rango de inductancias medibles sea $L_x\in[0.4\si{\milli\henry};
    2.1\si{\milli\henry}]$. Además, el rango de Q también debe ser $[0.25Q_N; Q_N]$.
    Estas características se deben cumplir cuando $f=10\si{\kilo\hertz}$ y $Q_N$ es
    el factor de calidad de la bobina patrón.
    
    \section{Diseño}
    \label{sec:ej3Design}
    Para la realización del puente, se tuvieron en cuenta las ecuaciones (\ref{eq:ej3V}),
    (\ref{eq:ej3L}), (\ref{eq:ej3Q}) y (\ref{eq:ej3R}).
    \begin{equation}
        V_d=\frac{(R_3 + \frac{1}{C_3})}{(R_1+R_3 + \frac{1}{C_3})} - \frac{R_4}{((\frac{1}{R_x}+\frac{1}{sL_x})^{-1}+R_4)} \times V_g
        \label{eq:ej3V}
    \end{equation}
    \begin{equation}
        L_x = C_3 R_1 R_4
        \label{eq:ej3L}
    \end{equation}
    \begin{equation}
        Q_x=\frac{1}{2 \pi f C_3 R_3}
        \label{eq:ej3Q}
    \end{equation}
    \begin{equation}
        R_x=\frac{R_1 R_4}{R_3}
        \label{eq:ej3R}
    \end{equation}

    Observando las ecuaciones (\ref{eq:ej3L}) y (\ref{eq:ej3Q}), se define que las variables
    de ajuste serán $R_1$ y $R_3$, dado que intentar variar $C_3$ sería demasiado 
    complicado y afecta a ambos $L_x$ y $Q_x$. Además, aunque $R_4$ también parece
    ser un candidato viable, es preferible mantenerlo constante para que la relación
    (\ref{eq:ej3A}) se mantenga constante.
    
    \begin{equation}
        A=\frac{Z_4}{Z_2}=\frac{Z_3}{Z_1}
        \label{eq:ej3A}
    \end{equation}
    
    Teniendo en cuenta las variables a ser modificadas, los márgenes de operación
    descriptos en la sección \ref{sec:ej3Specs} calcularon los valores para cada componente.
    De las expresiones anteriores se obtiene que

    \begin{equation}
        R_1 \in [\frac{L_{min}}{C_3 R_4};\frac{L_{MAX}}{C_3 R_4}] = [1.8182\si{\kilo\ohm}; 9.5455\si{\kilo\ohm}]
        \label{eq:ej3ranR1}
    \end{equation}

    \begin{equation}
        R_3 \in [\frac{1}{2 \pi f Q_{MAX}}; \frac{1}{2 \pi f Q_{min}}] = [136.50\si{\ohm}; 545.98\si{\ohm}]
        \label{eq:ej3ranR3}
    \end{equation}

    Dados los rangos obtenidos en las expresiones \ref{eq:ej3ranR1} y \ref{eq:ej3ranR3}, y considerando
    los componentes disponibles, se eligieron los valores en el cuadro \ref{tab:ej3Specs} para construir
    el puente y se obtiene el diseño en la Figura \ref{fig:ej3Design}.

    \begin{table}[h]
        \begin{center}
            \begin{tabular}{|c|c|}
                \hline
                Componente & Valor \\
                \hline
                $R_1$ & $[1.8\si{\kilo\ohm} \rightarrow 11.8\si{\kilo\ohm}]$\\
                $R_3$ & $[100\si{\ohm} \rightarrow 600\si{\ohm}]$\\
                $R_4$ & $100\si{\ohm}$\\
                $C_3$ & $2.2\si{\nano\farad}$\\
                \hline
            \end{tabular}
            \caption{Especificaciones elegidas para los componentes}
            \label{tab:ej3Specs}
        \end{center}
    \end{table}
    \begin{figure}[ht!]
        \begin{center}
            \begin{circuitikz}[scale = 0.75, transform shape, rotate = -90]
    \draw
    (0,0) to [sV, l=$V_g$] (0,14)
    (4,0)
    to [R , l=$100\si{\ohm}$] (4,2)
    to [vR, l=$500\si{\ohm}$] (4,4)
    to [C , l=$2.2\si{\nano\farad}$] (4,6)
    to [R , l=$1.8\si{\kilo\ohm}$] (4,8)
    to [vR, l=$100\si{\ohm}$] (4,10)
    to [vR, l=$1\si{\kilo\ohm}$] (4,12)
    to [vR, l=$20\si{\kilo\ohm}$] (4,14)
    (8,0)
    to [R , l=$100\si{\ohm}$] (8,6)
    to [L , l=$L_x$] (8,14)
    (8,8) to (7,8) to [R, l=$R_x$] (7,12) to (8,12)
    (0,14) to (8,14)
    (0,0) to (8,0)
    (4,6) to [voltmeter, l=$V_d$] (8,6)
    ;
\end{circuitikz}
\caption{Puente diseñado}
\label{fig:ej3Design}
        \end{center}
    \end{figure}

    Los resistores variables de menor tamaño fueron puestos con el propósito de reducir
    la sensibilidad del puente respecto a $R_1$.

    \newpage
    \section{Simulación y Sensibilidades}